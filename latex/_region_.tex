\message{ !name(code_generator.tex)}% from http://www.texample.net/tikz/examples/servers/
%
% A mindmap showing TeX online projects supported
% by DANTE e.V. which sponsors their server costs.
% Author: Stefan Kottwitz

\documentclass[border=10pt]{standalone}
%%%<
\usepackage{verbatim}
%%%>
\begin{comment}
  :Title: A mindmap showing TeX projects supported by DANTE e.V
  :Tags: Mindmaps
  :Author: Stefan Kottwitz
  :Slug: servers

  A mindmap showing TeX projects supported by DANTE e.V.,
  the german language TeX user group with home page
  at www.dante.de

  DANTE sponsors their server costs for the web sites
  shown in the mindmap.

  - TeX forums
  - TeX galleries
  - TeX blogs
  - Tools, documentation and FAQ

  Less for explanation, as the code has been done quick
  and far from perfect, but is shown here to share.
\end{comment}

\usepackage[utf8]{inputenc}
%% \usepackage{dtklogos}
\usepackage{tikz}
\usetikzlibrary{mindmap, shadows}
\usepackage[hidelinks,
  pdfauthor={Stefan Kottwitz},
  pdftitle={A mindmap showing TeX projects supported by DANTE e.V},
  pdfsubject={no subject},
  pdfkeywords={Mindmaps},
  pdfproducer={Latex with hyperref},
  pdfcreator={pdflatex},
  pdfencoding=auto]{hyperref}

% Information boxes
\newcommand*{\info}[4][16.3]{%
  \node [ annotation, #3, scale=0.65, text width = #1em,
    inner sep = 2mm ] at (#2) {%
    \list{$\bullet$}{\topsep=0pt\itemsep=0pt\parsep=0pt
      \parskip=0pt\labelwidth=8pt\leftmargin=8pt
      \itemindent=0pt\labelsep=2pt}%
    #4
    \endlist
  };
}

\begin{document}

\message{ !name(code_generator.tex) !offset(-3) }


\begin{tikzpicture}[ every annotation/.style = {draw,
      fill = white, font = \Large}]

  \path[mindmap,concept color=black!40,text=white,
    every node/.style={concept,circular drop shadow},
    %
    root/.style    = {concept color=black!40,
      font=\large\bfseries,text width=13em},
    %
    level 1 concept/.append
    style = {
      font=\large\bfseries,
      sibling angle=50,
      text width=7.7em,
      level distance=15em,
      inner sep=0pt
    },
    %
    level 2 concept/.append style={font=\bfseries,level distance=9em},
  ]
  %
  node[root] {Code Generator
    \begin{itemize}
    \item {\it automatisation des tâches de programmation répétitives}
    \item {\it respect des conventions de codage}
    \end{itemize}
  } [clockwise from=0]
  %
  child[concept color=blue!60] {
    node {\href{http://golatex.de}{go\LaTeX\\.de}} [clockwise from=90]
    child { node (goForum) {\href{http://golatex.de/index.html}{Forum}} }
    child { node (goWiki) {\href{http://golatex.de/wiki/Hauptseite}{Wiki}} }
  }
  %
  child[concept color=blue] {
    node[concept] {\href{http://texwelt.de}{\TeX welt\\.de}}
    [clockwise from=30]
    child { node[concept] (TeXnique)
      {\href{http://texnique.fr}{\TeX nique\\.fr}} }
    child { node[concept] (TeXweltQA)
      {\href{http://texwelt.de/wissen/}{Fragen~\& Antworten}} }
    child { node[concept] (TeXweltBlog)
      {\href{http://texwelt.de/blog/}{User blog} }}
  }
  child[concept color=green!40!black] {
    node[concept] {\href{http://texample.net/}{\TeX ample\\.net}}
    [clockwise from=310]
    child { node[concept] (TikZGalerie)
      {\href{http://texample.net/tikz/examples/}{TikZ-Galerie}} }
    child { node[concept] (TeXampleBlog)
      {\href{http://texample.net/weblog/}{Blog}} }
    child { node[concept] (Planet)
      {\href{http://texample.net/community/}{Planet}} }
  }
  child[concept color=red] {
    node[concept] (PGFPlots) {\href{http://pgfplots.net}{PGFPlots\\.net}}
    [clockwise from=270]
  }
  child[concept color=red!60!black] {
    node[concept] {\href{http://latex-community.org/}{\LaTeX-Community\\.org}}
    [counterclockwise from=100]
    child { node[concept] (LaTeXForum)
      {\href{http://latex-community.org/forum/}{Forum}}}
    child { node[concept] (LaTeXArtikel)
      {\href{http://latex-community.org/know-how}{Artikel-Archiv}} }
    child { node[concept] (LaTeXNews)
      {\href{http://latex-community.org/home/news}{News}} }
  }
  child[concept color=orange] {
    node[concept] (TeXdoc)
    {\href{http://texdoc.net/}{\TeX doc\\.net}}
    [clockwise from=100]
    child { node[concept] {\href{http://www.tex.ac.uk}{UK \TeX \\FAQ}}
  }}
  child[concept color=yellow!60!black] {
    node[concept] (Blogs) {Blogs} [clockwise from=139]
    child { node[concept] {\href{http://texblog.net/}{\TeX blog\\.net}}}
    child { node[concept] {\href{http://tikz.de/}{TikZ.de}} }
    child { node[concept] (Cookbook)
      {\href{http://latex-cookbook.net/}{\LaTeX-\\Cookbook\\.net}} }
  };
  \info{goForum.north east}{above,anchor=west,xshift=1em}{%
  \item[] Seit 2008
  \item 68\,444 Beiträge
  \item 13\,715 Themen
  \item 5\,532 registrierte Nutzer
  }
  \info{LaTeXForum.north west}{above,anchor=south}{%
  \item[] Seit 2008
  \item 81\,991 Beiträge
  \item 21\,026 Themen
  \item 13\,354 registrierte Nutzer
  }
  \info[8]{LaTeXArtikel.west}{below,anchor=north east,xshift=3em,yshift=-2em}{%
  \item 115 Artikel
  }
  \info[11]{LaTeXNews.south west}{below,anchor=north}{%
  \item 240 Meldungen
  }
  \info[9]{TikZGalerie.south}{below,anchor=north}{%
  \item[] Seit 2006
  \item 172 Autoren
  \item 384 Beispiele
  }
  \info[15]{goWiki.south}{below,anchor=north,xshift=3em}{%
  \item 152 erklärte Konzepte, Befehle und Pakete
  }
  \info{TeXweltQA.south east}{above,anchor=north west}{%
  \item[] Seit 2013
  \item 1\,710 Fragen
  \item 2\,151 Antworten
  \item 479 registrierte Nutzer
  }
  \info[8]{TeXweltBlog.south}{below,anchor=north,xshift=2em}{%
  \item[] Seit 2013
  \item 14 Autoren
  }
  \info[9]{PGFPlots.south west}{anchor=north east,xshift=1em}{%
  \item 14 Autoren
  \item 59 Beispiele
  }
  \info[6]{Planet.west}{anchor=east}{%
  \item 46 Blogs
  }
  \info[14]{TeXnique.east}{anchor=west,xshift = 0.5em}{%
  \item[] 2015, aufgrund Idee mit französischen
    \TeX-Freunden nach der TUG Damstadt, experimentell
  }
  \info[16]{Cookbook.east}{anchor=south west}{%
  \item[] Ab 10/2015, soll ca. 100 Beispiele aus
    dem \LaTeX\ Cookbook zeigen, sowie
    Community-Rezepte
  }
\end{tikzpicture}

\end{document}

\message{ !name(code_generator.tex) !offset(-200) }
